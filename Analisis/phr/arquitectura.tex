%========================================================
%Arquitectura de Proceso
%========================================================

%========================================================
%Revisión
%-------------------------------------------

% \UCccitem{Versión}{1}
% \UCccsection{Análisis de Procesos }
% \UCccitem{Autor}{nombreAutor}
% \UCccitem{Evaluador}{nombreEvaluador}
% \UCccitem{Prioridad}{Alta} %Alta, media, baja
% \UCccitem{Estatus}{} %Edición, Terminado, Corrección, Aprobado 
% \UCccitem{Complejidad}{Alta} %Alta, Media, Baja
% \UCccitem{Volatilidad}{Alta} %Alta, Media, Baja
% \UCccitem{Madurez}{Media}  %Alta, Media, Baja
% \UCccsection{Control de cambios}
% \UCccitem{Versión 0}{
% \begin{UClist}
% \RCitem{ Pxn T1}{Corregir la ortografía}{\DONE}
% \TODO es para solicitar un cambio, \TOCHK Es para informar que se atendió el TODO, \DONE Es para indicar que el evaluador reviso los cambios.
% \end{UClist}
%}

%========================================================
% Descripción general del proceso
%-----------------------------------------------
\begin{Arquitectura}{AG-RH}{Arquitectura General del Proceso de Registro de Horarios} {
		
			%-------------------------------------------
		%Descripción
		El proceso de Registro de Horarios se encarga de gestionar la generación y asignación de horarios para las unidades de aprendizaje\footnote{ver \refElem{UnidadDeAprendizaje}} a ofertar en el \refElem{ProgramaAcademico}. Para llevar a cabo cada una de sus funciones, se apoya de la interacción con otros procesos que involucran la obtención de información referente a los docentes adscritos a la \refElem{UnidadAcademica}, que incluyen horarios laborales y dictámenes de categorías, además del envío de información para validar la \refElem{EstructuraAcademica}. \\
		
		
		
		 
		%-------------------------------------------
		%Diagrama de arquitectura
		\noindent La Figura \cdtRefImg{arq:AG-RH}{AG-RH Diagrama de Arquitectura General del Proceso de Registro de Horarios} muestra la interacción con los procesos que proporcionan la información necesaria para llevar a cabo sus funciones.
		
		%\Pfig[0.8]{phr/imagenes/ArquitecturaPP-IR3Inscripcion_Reinscripcion}{arq:AG-IR3}{AG-IR3 Inscripción/Reinscripción}
		\Pfig[1]{phr/imagenes/ArquitecturaPP_RH_Arquitectura_Generacion_y_Asignacion_De_Horarios}{arq:AG-RH}{AG-RH Registro de Horarios}
		
	}{AG-RH}


\end{Arquitectura}

%========================================================
%Interacción
%-----------------------------------------------

\begin{ADescripcion}
	
	\item \textbf{PP-CH Capital Humano}. Este proceso envía la información referente a los docentes adscritos a la \refElem{UnidadAcademica}.

	\item \textbf{PG-EA Proceso de Estructura Académica}. Este proceso recibe la propuesta de horarios para las unidades de aprendizaje y envía a cada \refElem{UnidadAcademica} su aprobación y en su defecto su propuesta de cambio.
	
	
%%%%	este proceso erecibe la propuesta de horarios para las ua y envía a cada unidad academica su aprobaion y en su defecto su propuesta de cambio
	
%	\item \textbf{PG-EA Proceso de Estructura Académica}. Este proceso se encarga de evaluar la estructura académica, en caso de tener una observación solicitar la propuesta de reestructuración, enviar a cada unidad de académica las unidades de aprendizaje autorizadas a ofertar con horarios asignados, los cuáles serán enviados a Capital humano.
	
%	\item \textbf{PP-UA Unidad Académica}. El proceso se encarga de envíar a la Dirección de Educación Superior la propuesta de estructura académica para su evaluación o en su caso aprobación. 
	
	
\end{ADescripcion}

%\begin{PDescripcion}		
%\end{PDescripcion}