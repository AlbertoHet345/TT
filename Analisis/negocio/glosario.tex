\section{Glosario de términos}

	Esta sección describe de forma breve y sencilla los términos que son usados a lo largo del documento y que se consideran necesario para ayudar a comprender la jerga utilizada en el  CVH.


	
\begin{bGlosario}
	
	\bTerm{Alta}{Alta} Documento que acredita y reconoce la inscripción. %MP DAE
	
	\bTerm{AlumnoRegistradoEnSIRCEI}{Alumno Registrado en SIRCEI} Alumno del nivel medio superior o superior, reconocido en el Sistema de Registro y Control Escolar Institucional, para su control y seguimiento a su trayectoria académica, y consulta de calificaciones, vía internet. %MP DAE
	
	\bTerm{AntecedentesDeIngreso}{Antecedentes de Ingreso} Reseña de la Admisión del alumno en la Institución. % MP DAE
	
	\bTerm{AntecedentesDeTrayectoria}{Antecedentes de Trayectoria} Situación del alumno que ha llevado una dirección dentro del Instituto a lo largo del tiempo. % MP DAE
		
	\bTerm{AspiranteAceptado}{Aspirante Aceptado} Aspirante que al haber aprobado el examen de admisión y que cumple con los requisitos establecidos en la convocatoria correspondiente es asignado a una unidad académica. %Derivado
	
	
	\bTerm{AspiranteAExaminar}{Aspirante a Examinar}Persona que se registra en el proceso de admisión para ingresar a alguna de las escuelas, centros y unidades del Instituto, en los niveles que constituyen su oferta educativa. %MP DAE
	
	\bTerm{Boleta}{Boleta}Cédula de identidad que se otorga a los aspirantes, para que se inscriban como alumnos del Instituto Politécnico Nacional. %MP DAE
	
	\bTerm{Cadena}{Cadena} Es el Tipo de Dato definido por cualquier valor que se componga de una secuencia de caracteres, con o sin acentos, espacios, dígitos y signos de puntuación. Existen tres tipos de Cadenas: Palabra, Frase y Párrafo.
	
	\bTerm{CalendarioAcademico}{Calendario Académico} Programación que define los tiempos en los cuales se realizan anualmente las actividades académicas y de gestión escolar, en las diversas modalidades educativas que se imparten en el Instituto 
	
	\bTerm{CicloEscolar}{Ciclo Escolar} El lapso anual que define el calendario académico
	
	\bTerm{Convocatoria}{Convocatoria} Documento de carácter oficial emitido por una o varias Instituciones Educativas, en el cual se plantea tanto el procedimiento, los tiempos y fechas, los
	lineamientos legales en los que se basa la misma, y los requisitos que hay que cumplir para que un aspirante sea admitido al nivel de estudios superior inmediato concluido por él mismo, en la o las Instituciones que ofrecen sus distintas carreras y niveles de estudios impartidos por las mismas. % MP DAE
	
	\bTerm{CredencialDeAlumno}{Credencial de Alumno} Identificación oficial expedida por la Dirección de Administración Escolar del IPN, para todos aquellos alumnos que cursan alguna carrera en uno de los planteles educativos del Instituto, ya sea a Nivel Medio Superior o Superior. %MP DAE
	
	\bTerm{DictamenDeCategoria}{Dictámen de Categoría} Documento que determina la categoría a la que corresponde un profesor de acuerdo a sus méritos académicos y profesionales. Es expedido por la  Comisión Mixta Paritaria de Evaluación de Categoría Docente. % MANUAL PARA DOCENTES Y COORDINADORES DEL PROCESO DE EVALUACIÓN DE CATEGORÍA DOCENTE
	
	
	\bTerm{Entero}{Entero} Es el Tipo de Dato definido por todos los valores numéricos enteros, tanto positivos como negativos.
	
	\bTerm{EstructuraAcademica}{Estructura Académica} Al conjunto de grupos, horarios y unidades de aprendizaje organizadas para el período siguiente.
	
	\bTerm{ETS}{Examen a Título De Suficiencia} Es la evaluación que comprende el total de los contenidos del programa de estudios y que el alumno podrá presentar cuando no haya acreditado de manera ordinaria o extraordinaria alguna unidad de aprendizaje.
	
	\bTerm{ExpedienteAcademico}{Expediente Académico} Al documento que contiene la información y el historial académico del alumno.
	
	\bTerm{ExpedienteDelAspirante}{Expediente del Aspirante} Se refiere a la documentación que el aspirante entrega durante el proceso de admisión para su verificación. % Derivado
	
	\bTerm{Fecha}{Fecha} Es el Tipo de Dato definido por todas las fecha pasadas y futuras. Se representa de dos formas: Fecha Corta y Fecha Larga.
	
	\bTerm{FechaCorta}{Fecha corta} Representación de una fecha de la forma DD/MM/YYYY, ejemplo: 24/02/2012.
	
	\bTerm{FechaLarga}{Fecha Larga} Representación de una fecha de la forma DD de MMMM, del YYYY, ejemplo: 24 de Febrero, del 2012.
	
	\bTerm{Frase}{Frase} Cadena formada por mas de una palabra y que puede ocupar hasta un par de renglones.
	
	\bTerm{HistorialAcademico}{Historial Académico} Se entiende por historial académico al conjunto de calificaciones obtenidas por un alumno en las unidades de aprendizaje,durante su vida académica dentro del Instituto.
	
	\bTerm{HojaDeResultadoDelExamenDeAdmision}{Hoja de Resultado del Examen de Admisión} Documento que el interesado imprime a través de internet, para conocer la opción educativa donde quedó aceptado. Con la Hoja de Resultados los aspirantes seleccionados obtendrán una cita donde podrán continuar con sus trámites de inscripción al Instituto. %MP DAE
	
	\bTerm{Horario}{Horario} Documento que se le otorga a un aspirante o alumno en el que se confirma su inscripción al período en curso.
	
	\bTerm{Incidencia}{Incidencia}Acontecimiento que sobreviene en el curso de un asunto o negocio y tiene con él alguna conexión.
	
	
	\bTerm{Kardex}{Kardex} Documento donde se resgistran los datos personales del alumno, el cual contiene la trayectoria escolar mediante las calificaciones obtenidas en las asignaturas y/o unidades de aprendizaje cursadas, de acuerdo a un plan de estudios, formas de evaluación y periodos escolares. % CIRCULAR NO5 Criterios de Expedición de Boletas de calificaciones
	
	\bTerm{Modalidad}{Modalidad Educativa} Forma en que se organizan, distribuyen y desarrollan los planes y programas de estudio para su impartición. Existen 3 tipos de modalidades:
	\begin{itemize}
		\item [Escolarizada:] La que se desarrolla en aulas, talleres, laboratorios y otros ambientes de aprendizaje, en horarios y periodos determinados.
		\item [No Escolarizada:] Es la que se desarrolla fuera de aulas,
		talleres, laboratorios y no necesariemente comprende horarios determinados.
		\item [Mixta:] Es la combinación de modalidades educativas de
		acuerdo con el diseño un programa académico en particular.
	\end{itemize}
	
	\bTerm{Nombramiento}{Nombramiento} Proceso por el cual un aspirante a dar cátedra en el instituto es elegido otorgándole las horas que deberá cumplir. % RCITPAIPN
	
	
	\bTerm{OficioDeAlumnoAceptado}{Oficio de Alumno Aceptado} Documento generado por la Dirección de Administración Escolar del IPN, por medio del cual el aspirante es notificado que al concluir los trámites de inscripción adquiere el estatus de alumno del Instituto. % MP DAE
	
	\bTerm{PeriodoEscolar}{Periodo Escolar} Lapso señalado en el calendario académico para cursar unidades de aprendizaje de un programa académico.
	
	\bTerm{PlanDeEstudios}{Plan de Estudios} Estructura curricular que se deriva de un programa académico y que permite cumplir con los propósitos de formación general,
	la adquisición de conocimientos y el desarrollo de capacidades correspondientes a un nivel y modalidad educativa.%Reglamento General de Estudios
	
	
	\bTerm{Preboleta}{Preboleta} Es un numéro de identificación que se le asgina al aspirante como fin de control interno durante el proceso de admisión. %Derivado
	
	\bTerm{ProcesoDeAdmision}{Proceso de Admisión} Conjunto de etapas que deben ser realizadas tanto por la Institución educativa como por el aspirante. Comprende desde la publicación de la convocatoria, el regist ro de aspirantes, la aplicación del examen de admisión, publicación de resultados e inscripciones de aspirantes seleccionados. %MP DAE
	
	\bTerm{ProgramaAcademico}{Programa Académico} Al conjunto organizado de elementos necesarios para generar, adquirir y aplicar el conocimiento en un campo específico;así como para desarrollar
	habilidades, actitudes y valores en el alumno, en
	diferentes áreas del conocimiento. %Reglamento General de Estudios
	
	\bTerm{RCITPAIPN}{Reglamento de  las  Condiciones Interiores de Trabajo del Personal Académico del IPN} Reglamento que fija las condiciones de trabajo del personal académico del
	Instituto Politécnico Nacional, que conjuntamente con sus tres anexos:
	\begin{itemize}
		\item[I.] Prestaciones Sociales y Económicas.
		\item[II.] Seguridad e Higiene.
		\item[III.] Promoción Docente.
	\end{itemize}
	Son de observancia obligatoria para el personal académico, el titular y demás funcionarios del Instituto
	Politécnico Nacional y de sus Organos de Apoyo.
	
	\bTerm{RUAA}{Registro Único de Actividades Académicas } Es  el  documento  que  avala  las  funciones  y  actividades  que  desarrolla  el  docente
	
	\bTerm{TrayectoriaEscolar}{Trayectoria Escolar} Al proceso a través del cual el alumno construye su formación con base en un plan de estudio.
	
	\bTerm{UnidadDeAprendizaje}{Unidad de Aprendizaje} A la estructura didáctica que integra los contenidos formativos de un curso, materia, módulo, asignatura o sus equivalentes.
	En general, las unidades de aprendizaje deberán cursarse y acreditarse conforme lo establezca el plan de estudio, y podráan seleccionarse de entre
	la oferta disponible en el periodo escolar y sujeta a grupo. %Reglamento General de Estudios
	
	\bTerm{ValidacionDeInscripcion}{Validación de Inscripción} Proceso a través del cual se dictamina la autenticidad y legitimidad de los documentos aportados por el aspirante para su inscripción, si la documentación es correcta se le comunica por escrito. %MP DAE
	
	
	
	% TODO:Agregar fuente
	
 
\end{bGlosario}



