%-------------------MODELO ESTRUCTURAL--------------------------------------
\section{Modelo Estructural del Negocio} 

%TODO: redactar sección, realizar el diagrama de clases de la estructura de la información, y restriñir atributos y relaciones de cada Entidad (Clase).

	En esta sección se explica con detalle la estructura de cada elemento del negocio enfatizando la forma en que se encuentran relacionados.
	Aquí se definen las Entidades especificando los atributos que las componen y  las relaciones entre ellas. La figura \ref{fig:clases} esquematiza la organización de las clases mediante paquetes de clases. Dado que esta versión es parte del entregable 1 del prototipo I, se muestra en color verde el paquete que pertenece al entregable 1 .\\

\begin{figure}[htbp]
  \begin{center}
    \includegraphics[width=.5\textwidth]{images/diagramas/ClasesCVH.png}
    \caption{Diagrama de Paquetes de Clases.}
    \label{fig:clases}
  \end{center}
\end{figure}

\newpage
\section{Registro de Usuario} 

	Como se mencionó anteriormente, el CVH está organizado en paquetes que agrupan clases asociadas a la funcionalidad del mismo paquete. En la figura \ref{clases-registroUsuarios} se muestra la interacción de las clases con las que se construirá el paquete de Registro de Usuarios y el Domicilio Geográfico.

\begin{figure}[htbp]
  \begin{center}
    \includegraphics[width=0.85\textwidth]{images/diagramas/ClassDiagramCVH.png}
    \caption{Diagrama de clases del Registro de Usuarios y domicilio Geográfico}
    \label{clases-registroUsuarios}
  \end{center}
\end{figure}





%=======================   PERSONA  ===========================
\begin{BusinessEntity}{persona}{Persona}
  \Battr{curp}{CURP}{Palabra}{Es la Clave Única de Registro de Población}{\obligatorio}
  \Battr{nombre}{Nombre}{Palabra}{Es el nombre de pila de la persona}{\obligatorio}
   \Battr{primerApellido}{Primer Apellido}{Palabra}{Es el primer apellido de la persona}{\obligatorio}
   \Battr{segundoApellido}{Segundo Apellido}{Palabra}{Es el segundo apellido de la persona}{\obligatorio}
   \Battr{fechaNacimiento}{Fecha de Nacimiento}{Palabra}{Es la fecha de nacimiento de la persona}{\obligatorio}
    \Battr{genero}{Género}{Palabra}{Es el género de la persona}{\obligatorio}
 \end{BusinessEntity}   

\subsubsection{Relaciones}

\begin{BussinesFact}{Información Adicional de una Persona}
  \BRitem{Descripción}{Una \refEnt{persona}{Persona} tiene \refEnt{personaInfoAdicional}{Información Adicional}.}
  \BRitem{Tipo}{Composición}
  \BRitem{Cardinalidad}{uno a uno}
\end{BussinesFact}

\begin{BussinesFact}{Cuenta de una Persona}
  \BRitem{Descripción}{Una \refEnt{persona}{Persona} tiene asociada una \refEnt{cuenta}{Cuenta} para permitir el acceso al sistema.}
  \BRitem{Tipo}{Composición}
  \BRitem{Cardinalidad}{uno a uno}
\end{BussinesFact}

\begin{BussinesFact}{País de Origen de una Persona}
  \BRitem{Descripción}{Las \refEnt{persona}{Personas} tienen un \refEnt{paisOrigen}{País de Origen}.}
  \BRitem{Tipo}{Composición}
  \BRitem{Cardinalidad}{muchos a uno}
\end{BussinesFact}

\begin{BussinesFact}{Medios de Contacto de una Persona}
  \BRitem{Descripción}{Las \refEnt{persona}{Personas} tienen \refEnt{mediosContacto}{Medios de Contacto}.}
  \BRitem{Tipo}{Composición}
  \BRitem{Cardinalidad}{uno a muchos}
\end{BussinesFact}

%=======================   PERSONA  INFORMACION ADICIONAL ===========================
\begin{BusinessEntity}{personaInfoAdicional}{Persona Información Adicional}    
	\Battr{rfc}{RFC}{Palabra}{Es el Registro Federal de Contribuyentes de la persona}{\obligatorio}
	\Battr{edoCivil}{Estado Civil}{Palabra}{Es el estado civil de la persona}{\obligatorio}
 \end{BusinessEntity} 
 
% \begin{BussinesFact}{Información Adicional de una Persona}
%  \BRitem{Descripción}{La \refEnt{personaInfoAdicional}{Información Adicional de una Persona} esta asociada una \refEnt{persona}{Persona}.}
%  \BRitem{Tipo}{Composición}
%  \BRitem{Cardinalidad}{uno a uno}
%\end{BussinesFact}
 
 
%=======================   CUENTA  ===========================
\begin{BusinessEntity}{cuenta}{Cuenta}
     \Battr{idUsuario}{IdUsuario}{Palabra}{Es el identificador de usuario para poder ingresar al sistema, puede ser el mismo que peoplesoft}{\obligatorio}
     \Battr{contrasena}{Contraseña}{Palabra}{Es la contraseña  para poder ingresar al sistema}{\obligatorio}
          \Battr{idPeople}{Id Peoplesoft}{Palabra}{Es el identificador bajo el estándar de peoplesoft}{\obligatorio}
     \Battr{activa}{Activa}{boolean}{Es para identificar si la cuenta se encuentra activa o inactiva}{\obligatorio}
     \Battr{validacionRENAPO}{validación RENAPO}{boolean}{Es el identificador que define si la información personal del usuario se encuentra validada con RENAPO o no}{\obligatorio}
\end{BusinessEntity}

\subsubsection{Relaciones}
\begin{BussinesFact}{Token de una Cuenta}
  \BRitem{Descripción}{Una \refEnt{cuenta}{Cuenta} tiene asociado \refEnt{token}{Tokens} para manejar saber si ya se ha intentado activar o cancelar la cuenta.}
  \BRitem{Tipo}{Composición}
  \BRitem{Cardinalidad}{uno a muchos}
\end{BussinesFact}

%=======================   TOKEN ===========================
\begin{BusinessEntity}{token}{Token}    
	\Battr{hash}{Hash}{Palabra}{Es una cadena que tiene el valor del token}{\obligatorio}
	\Battr{creacion}{Creación}{Fecha}{Es la fecha en que se creó el token}{\obligatorio}
	\Battr{duracion}{Duración}{Numérico}{Es el número en días que estará activo el token}
	\Battr{estatus}{Estatus}{Boolean}{Define si el token esta activo o cancelado. Será verdadero cuando no ha sido utilizado el token}{\obligatorio}
	\Battr{observaciones}{Observaciones}{Párrafo}{Es la descripción textual sobre el motivo de su cancelación o reestablecimiento}{\obligatorio}
 \end{BusinessEntity} 

\subsubsection{Relaciones}


\begin{BussinesFact}{Tipo de Token de un Token}
  \BRitem{Descripción}{Un \refEnt{token}{Token}  tiene un  \refEnt{tipoToken}{Tipo de Token} asociado.}
  \BRitem{Tipo}{Composición}
  \BRitem{Cardinalidad}{uno a uno}
\end{BussinesFact}

%=======================   TIPO DE TOKEN  ===========================
\begin{BusinessEntity}{tipoToken}{Tipo de Token}    
	\Battr{tipo}{Tipo}{Numérico}{Es el tipo de token asociado al Token, que puede ser activación de cuenta, cancelar activación, reestablecer contraseña de acceso}{\obligatorio}
 \end{BusinessEntity} 


%=======================   PAIS DE ORIGEN ===========================
\begin{BusinessEntity}{paisOrigen}{País de Origen}    
	\Battr{nombre}{Nombre}{Palabra}{Es el nombre del país de origen}{\obligatorio}
 \end{BusinessEntity} 
 
%  \begin{BussinesFact}{País de Origen de una Persona}
%  \BRitem{Descripción}{Los \refEnt{paisOrigen}{Paises de Orígen} están asociados a una \refEnt{persona}{Persona}.}
%  \BRitem{Tipo}{Composición}
%  \BRitem{Cardinalidad}{muchos a uno}
%\end{BussinesFact}

%=======================   MEDIOS DE CONTACTO ===========================
\begin{BusinessEntity}{mediosContacto}{Medios de Contacto}    
	\Battr{valor}{Valor del Contacto}{Palabra}{Es el valor de define al contacto, por ejemplo 5523124567 para un número telefónico o jperez@mail.com para un correo electrónico}{\obligatorio}
	\Battr{primario}{primario}{boolean}{Es para identificar si el contacto es primario o no}{\obligatorio}
 \end{BusinessEntity} 

\subsubsection{Relaciones}

%\begin{BussinesFact}{Teléfono de un Medio de Contacto}
%  \BRitem{Descripción}{Los \refEnt{mediosContacto}{Medios de Contacto}  le pertenecen a una  \refEnt{persona}{Persona}.}
%  \BRitem{Tipo}{Composición}
%  \BRitem{Cardinalidad}{muchos a uno}
%\end{BussinesFact}


\begin{BussinesFact}{Teléfono de un Medio de Contacto}
  \BRitem{Descripción}{Un \refEnt{mediosContacto}{Medio de Contacto}  puede ser un  \refEnt{telefono}{Teléfono}, el teléfono tiene la característica que puede registrar el número de extensión.}
  \BRitem{Tipo}{Generalización}
  \BRitem{Cardinalidad}{uno a uno}
\end{BussinesFact}


\begin{BussinesFact}{Tipo de Medio de Contacto}
  \BRitem{Descripción}{Un \refEnt{mediosContacto}{Medio de Contacto}  tiene un  \refEnt{tipoComunicacion}{Tipo de Contacto}.}
  \BRitem{Tipo}{Composición}
  \BRitem{Cardinalidad}{uno a muchos}
\end{BussinesFact}

%=======================   TELEFONO ===========================
\begin{BusinessEntity}{telefono}{Teléfono}    
	\Battr{extension}{Extensión}{Número}{Es la extensión del teléfono}{\opcional}
 \end{BusinessEntity} 
 
% \subsubsection{Relaciones}
% 
% \begin{BussinesFact}{Medio de Contacto tipo Teléfono}
%  \BRitem{Descripción}{Un \refEnt{telefono}{Teléfono} es un  \refEnt{mediosContacto}{Medio de Contacto}.}
%  \BRitem{Tipo}{Especialización}
%  \BRitem{Cardinalidad}{uno a uno}
%\end{BussinesFact}
 

%=======================   TIPOS DE CONTACTO  ===========================
\begin{BusinessEntity}{tipoContacto}{Tipos de Contacto}    
	\Battr{nombre}{Nombre}{Palabra}{Es el tipo de Contacto}{\obligatorio}
	\Battr{icono}{Icono}{Palabra}{Es la referencia relativa a la ubicación del icono que representa el tipo de contacto}{\obligatorio}
 \end{BusinessEntity} 

% \subsubsection{Relaciones}
% 
% \begin{BussinesFact}{Tipos de Medios de Contacto}
%  \BRitem{Descripción}{Un \refEnt{tipoContacto}{Tipo de Contacto} está asociado a un  \refEnt{mediosContacto}{Medio de Contacto}.}
%  \BRitem{Tipo}{Composición}
%  \BRitem{Cardinalidad}{muchos a uno}
%\end{BussinesFact}

\newpage
\section{Domicilio Geográfico} 

	En esta sección se esquematiza el contenido del paquete de clases \textit{Domicilio Geográfico} mostrado en la figura \ref{fig:clases} las clases que operarán en el sistema y que interactuarán con las del \textit{Registro de Usuario}.


%=======================   DOMICILIO GEOGRÁFICO   ===========================
%
\begin{BusinessEntity}{domicilioGeografico}{Domicilio Geográfico}  
	\Battr{tipoDomicilio}{Tipo}{Tipo de Domicilio}{Es el tipo de domicilio Geográfico que posee la Persona. Los tipos son: Urbano, Rural o en Vías de comunicación}{\obligatorio}
	\Battr{localidad}{Localidad}{Localidad}{Especifica la Localidad, Municipio y Entidad federativa en dónde se encuentra ubicado el domicilio de la Persona}{\obligatorio}
	\Battr{coordenada}{Coordenadas Geográficas}{PuntoWGS84}{Coordenada Geográfica del domicilio de la Persona}
	\Battr{vialidad}{Vialidad Principal}{Vialidad}{Es la Vialidad sobre la que esta establecido el domicilio de la Persona}{\obligatorio}
	\Battr{numeroExterior}{Número Exterior}{Número y Letra}{Numeración al exterior de el inmueble o el valor S/N en caso de que no se conozca la numeración}{\obligatorio}
	\Battr{numeroInterior}{Número Interior}{Número y Letra}{Numeración al interior del inmueble}{\obligatorio}
	\Battr{numeroExteriorAnterior}{Número Exterior Anterior}{Número y Letra}{Numeración que existía en la dirección antes de la actual}{\obligatorio}
	\Battr{asentamiento}{Asentamiento}{Asentamiento}{Es la información del asentamiento del domicilio de la Persona}{\obligatorio}
	\Battr{cp}{Código Postal}{Cinco dígitos}{Código Postal asignado por Correos de México}{\obligatorio}
	\Battr{referenciaVialidadA}{Referencia Vialidad A}{Vialidad}{Es la vialidad anterior a la dirección}{\obligatorio}
	\Battr{referenciaVialidadB}{Referencia Vialidad B}{Vialidad}{Es la vialidad posterior a la dirección}{\obligatorio}
	\Battr{vialidadPosterior}{Vialidad Posterior}{Vialidad}{Es la Vialidad siguiente a la Vialidad Principal}{\obligatorio}
\end{BusinessEntity}

%\subsubsection{Relaciones}
%\begin{BussinesFact}{Entidad Federativa del Centro De Trabajo}
%	\BRitem{Descripción}{El \hyperlink{centroDeTrabajo}{Centro De Trabajo} puede tener o no tener el registro de su \hyperlink{domicilioGeografico}{Domicilio Geográfico}.}
%	\BRitem{Tipo}{Composición}
%	\BRitem{Cardinalidad}{uno a uno}
%\end{BussinesFact}
%
%\begin{BussinesFact}{Municipio del Centro De Trabajo}
%	\BRitem{Descripción}{Todos los \refEnt{centroDeTrabajo}{Centros De Trabajo} deben tener por lo menos un \refEnt{servicio}{Servicio} registrado.}
%	\BRitem{Tipo}{Composición}
%	\BRitem{Cardinalidad}{muchos a muchos}
%\end{BussinesFact}
%\begin{BussinesFact}{Localidad del Centro de Trabajo}
%	\BRitem{Descripción}{Un \refEnt{centroDeTrabajo}{Centro De Trabajo} podrá registrarse sin estar asociado a un \refEnt{aula}{Aula}, sin embargo deberá tener al menos un \refEnt{aula}{Aula} registrada para poder registrar un grupo.}
%	\BRitem{Tipo}{Composición}
%	\BRitem{Cardinalidad}{uno a muchos}
%\end{BussinesFact}

%\begin{BussinesFact}{Domicilio del Centro de Trabajo}
%	\BRitem{Descripción}{Un \refEnt{centroDeTrabajo}{Centro De Trabajo} podrá registrase sin tener \refEnt{grupo}{Grupos}, sin embargo deberá tener al menos uno para poder inscribir alumnos.}
%	\BRitem{Tipo}{Composición}
%	\BRitem{Cardinalidad}{uno a muchos}
%\end{BussinesFact}

%\begin{BussinesFact}{Asentamiento del Centro de Trabajo}
%	\BRitem{Descripción}{Un \refEnt{centroDeTrabajo}{Centro De Trabajo} podrá registrarse sin tener \refEnt{docente}{Docentes}, sin embargo deberá tener al menos uno para poder registrar un curso.}
%	\BRitem{Tipo}{Composición}
%	\BRitem{Cardinalidad}{muchos a muchos}
%\end{BussinesFact}
%%=======================   ASENTAMIENTO    ===========================

\begin{BusinessEntity}{asentamiento}{Asentamiento}
	\Battr{tipoAsentamiento}{Tipo}{Tipo de Asentamiento}{Es el Tipo De Asentamiento sobre el cual esta ubicado el domicilio de la Persona}{\obligatorio}
	\Battr{nombre}{Nombre}{Palabra}{Es el nombre que identifica el Asentamiento}{\obligatorio}
\end{BusinessEntity}

%==============================CAMINO=======================================
\begin{BusinessEntity}{camino}{Camino}
	\Battr{nombre}{Nombre}{Frase}{Nombre con el que se conoce el camino}{\obligatorio}
	\Battr{origen}{Origen}{Frase}{Origen del camino}{\obligatorio}
	\Battr{destino}{Destino}{Frase}{Destino del camino}{\obligatorio}
	\Battr{cadenamiento}{Cadenamiento}{Numérico}{Numero de kilómetros recorridos desde el origen del camino para llegar al Domicilio Geográfico}{\obligatorio}
	\Battr{margen}{Margen}{Derecha/Izquierda}{Lado del camino en el que se encuentra el domicilio de la Persona}{\obligatorio}
	\Battr{terminoGenerico}{Término Genérico}{Termino Genérico}{Nombre genérico del camino}{\obligatorio}
	\end{BusinessEntity}

%============================CARRETERA==============================
\begin{BusinessEntity}{carretera}{Carretera}
	\Battr{nombre}{Nombre}{Frase}{Nombre oficial de la carretera}{\obligatorio}
	\Battr{origen}{Origen}{Frase}{Origen de la carretera}{\obligatorio}
	\Battr{destino}{Destino}{Frase}{Destino de la carretera}{\obligatorio}
	\Battr{cadenamiento}{Cadenamiento}{Numérico}{Numero de kilómetros recorridos desde el origen de la carretera para llegar al Domicilio Geográfico}{\obligatorio}
	\Battr{margen}{Marge}{Derecha/Izquierda}{Lado de la carretera en el que se encuentra el Usuario}{\obligatorio}
	\Battr{terminoGenerico}{Término Genérico}{Termino Genérico}{Es el Término genérico ``Carretera''}{\obligatorio}
	\Battr{administración}{Administración}{Administración de Carretera}{Especificación de quien administra la carretera}{\obligatorio}
	\Battr{derechoTransito}{Derecho de Tránsito}{Derecho de Tránsito}{Tipo de derecho para transitar que se tiene sobre la carretera}{\obligatorio}
\end{BusinessEntity}

%=======================    ENTIDAD FEDERATIVA    =================================

\begin{BusinessEntity}{entidadFederativa}{Entidad Federativa}
	\Battr{nombre}{Nombre}{Frase}{Es el nombre que identifica la Entidad Federativa}{\obligatorio}

\end{BusinessEntity}

\subsubsection{Relaciones}
\begin{BussinesFact}{Entidad Federativa de un Localidad}
	\BRitem{Descripción}{La \refEnt{entidadFederativa}{Entidad Federativa} donde se encuentra ubicado el domicilio o el lugar de nacimiento de la \refEnt{persona}{Persona}  será utilizado al momento de registrar una \refEnt{localidad}{Localidad} y la información de la \refEnt{persona}{Persona}.}
	\BRitem{Tipo}{Composición}
	\BRitem{Cardinalidad}{uno a muchos}
\end{BussinesFact}

%=======================    LOCALIDAD    ==============================


\begin{BusinessEntity}{localidad}{Localidad}
	\Battr{entidadFederativa}{Entidad Federativa}{Es la \refEnt{entidadFederativa}{Entidad Federativa} donde se encuentra ubicado el domicilio de la \refEnt{persona}{Persona}}{\obligatorio}
	\Battr{Municipio}{Municipio}{Es el \refEnt{municipio}{Municipio} donde se encuentra ubicado el domicilio de la \refEnt{persona}{Persona}}{\obligatorio}
	\Battr{nombre}{Nombre}{Frase}{Es el nombre que identifica la Localidad}{\obligatorio}

\end{BusinessEntity}

\subsubsection{Relaciones}
\begin{BussinesFact}{Entidad Federativa de una Localidad}
	\BRitem{Descripción}{Toda \refEnt{localidad}{Localidad} debe pertenecer a una \refEnt{entidadFederativa}{Entidad Federativa} de la Republica Mexicana.}
	\BRitem{Tipo}{Composición}
	\BRitem{Cardinalidad}{muchos a uno}
\end{BussinesFact}

\begin{BussinesFact}{Localidad asociada a un Municipio}
	\BRitem{Descripción}{Toda \refEnt{localidad}{Localidad} debe pertenecer a un \refEnt{municipio}{Municipio} de un Estado.}
	\BRitem{Tipo}{Composición}
	\BRitem{Cardinalidad}{muchos a uno}
\end{BussinesFact}

\begin{BussinesFact}{Localidad de un Domicilio Geográfico}
	\BRitem{Descripción}{La \refEnt{localidad}{Localidad} donde se encuentra ubicado el domicilio de la \refEnt{persona}{Persona} será utilizada al momento de registrar la ubicación geográfica por el \refEnt{domicilioGeografico}{Domicilio Geográfico}.}
	\BRitem{Tipo}{Composición}
	\BRitem{Cardinalidad}{uno a uno}
\end{BussinesFact}

%===============================LUGAR DE NACIMIENTO======================
\begin{BusinessEntity}{LugarNacimiento}{Lugar de Nacimiento}
	\Battr{pais}{País}{Frase}{Nombre del Pais del lugar de nacimiento}{\obligatorio}
	\Battr{EntidadFederativa}{Entidad Federativa}{Frase}{Nombre de la Entidad Federativa del lugar de nacimiento}{\obligatorio}
	\Battr{Municipio}{Municipio}{Frase}{Nombre del Municipio del lugar de nacimiento}{\obligatorio}
	\Battr{Localidad}{Localidad}{Frase}{Nombre de la Localidad del lugar de nacimiento}{\obligatorio}
\end{BusinessEntity}

\subsubsection{Relaciones}
\begin{BussinesFact}{Localidad de un Lugar de Nacimiento}
	\BRitem{Descripción}{Un \refEnt{LugarNacimiento}{Lugar de Nacimiento} se encuentra ubicado en una \refEnt{LugarNacimiento:Localidad}{Localidad}}
	\BRitem{Tipo}{Asociación}
	\BRitem{Cardinalidad}{muchos a uno}
\end{BussinesFact}


%=======================    MUNICIPIO    =================================



\begin{BusinessEntity}{municipio}{Municipio} 
	\Battr{nombre}{Nombre}{Frase}{Es el nombre que identifica el Municipio de un Estado}{\obligatorio}
\end{BusinessEntity}

\subsubsection{Relaciones}
\begin{BussinesFact}{Municipio de una Localidad}
	\BRitem{Descripción}{El \refEnt{municipio}{Municipio} donde se encuentra ubicado el domicilio de la \refEnt{persona}{Persona}, será utilizado por la \refEnt{localidad}{Localidad}.}
	\BRitem{Tipo}{Composición}
	\BRitem{Cardinalidad}{uno a muchos}
\end{BussinesFact}

%=======================   TIPO DE ASENTAMIENTO    ===========================

\begin{BusinessEntity}{tipoAsentamiento}{Tipo De Asentamiento}
	\Battr{tipo}{Tipo de Asentamiento}{Tipo de asentamiento}{Es el Tipo De Asentamiento sobre el cual está ubicado el domicilio de la \refEnt{persona}{Persona}}{\obligatorio}
	\Battr{nombre}{Nombre}{Palabra}{Es el nombre que identifica el Asentamiento}{\obligatorio}
\end{BusinessEntity}

\subsubsection{Relaciones}
\begin{BussinesFact}{Tipo De Asentamiento del domicilio de la Persona}
	\BRitem{Descripción}{Un \refEnt{tipoAsentamiento}{Tipo De Asentamiento} es utilizado al momento de ser registrado uno o más \refEnt{asentamiento}{Asentamientos}.}
	\BRitem{Tipo}{Composición}
	\BRitem{Cardinalidad}{Uno a muchos} 
\end{BussinesFact}

%=======================   TIPO DE DOMICILIO    ===========================

\begin{BusinessEntity}{tipoDomicilio}{Tipo De Domicilio}
	\Battr{descripcion}{Descripción}{Palabra}{Es el nombre que identifica el Tipo De Domicilio}{\obligatorio}

\end{BusinessEntity}

\subsubsection{Relaciones}
\begin{BussinesFact}{Tipo De Domicilio de la Persona}
	\BRitem{Descripción}{Un \refEnt{tipoAsentamiento}{Tipo De Domicilio} es utilizado al momento de ser registrado el  \refEnt{domicilioGeografico}{Domicilio Geográfico}.}
	\BRitem{Tipo}{Composición}
	\BRitem{Cardinalidad}{Uno a muchos} 
\end{BussinesFact}

%=======================   VIALIDAD    ===========================

\begin{BusinessEntity}{vialidad}{Vialidad}
	\Battr{tipoVialidad}{Tipo De Vialidad}{Tipo de Vialidad}{Es el Tipo De Vialidad correspondiente a la Vialidad donde se encuentra el Usuario}{\obligatorio}
	\Battr{nombre}{Nombre}{Frase}{Es el nombre que identifica la Vialidad}{\obligatorio}
%	\Battr{}{}{}{}{}
\end{BusinessEntity}
\subsubsection{Relaciones}
\begin{BussinesFact}{Tipo De Domicilio de la Persona}
	\BRitem{Descripción}{Un \refEnt{tipoAsentamiento}{Tipo De Domicilio} es utilizado al momento de ser registrado el  \refEnt{domicilioGeografico}{Domicilio Geográfico}.}
	\BRitem{Tipo}{Composición}
	\BRitem{Cardinalidad}{uno a muchos}
\end{BussinesFact}




%=======================    PAÍS    =================================
\begin{BusinessEntity}{país}{País}
	\Battr{idpais}{idPais}{Entero}{Clave asignada al país}{\obligatorio}
	\Battr{nombre}{Nombre}{Frase}{Nombre del país}{\obligatorio}
%	\Battr{}{}{}{}{}
\end{BusinessEntity}


