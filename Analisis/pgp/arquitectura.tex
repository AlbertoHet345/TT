%========================================================
%Arquitectura de Proceso
%========================================================

%========================================================
%Revisión
%-------------------------------------------

% \UCccitem{Versión}{1}
% \UCccsection{Análisis de Procesos }
% \UCccitem{Autor}{nombreAutor}
% \UCccitem{Evaluador}{nombreEvaluador}
% \UCccitem{Prioridad}{Alta} %Alta, media, baja
% \UCccitem{Estatus}{} %Edición, Terminado, Corrección, Aprobado 
% \UCccitem{Complejidad}{Alta} %Alta, Media, Baja
% \UCccitem{Volatilidad}{Alta} %Alta, Media, Baja
% \UCccitem{Madurez}{Media}  %Alta, Media, Baja
% \UCccsection{Control de cambios}
% \UCccitem{Versión 0}{
% \begin{UClist}
% \RCitem{ Pxn T1}{Corregir la ortografía}{\DONE}
% \TODO es para solicitar un cambio, \TOCHK Es para informar que se atendió el TODO, \DONE Es para indicar que el evaluador reviso los cambios.
% \end{UClist}
%}

%========================================================
% Descripción general del proceso
%-----------------------------------------------
\begin{Arquitectura}{AG-GP}{Arquitectura General del Proceso de Gestión de Profesores} {
		
		%-------------------------------------------
		%Descripción
		El proceso de gestión de profesores permite obtener la información de los docentes para la conformación de la \refElem{EstructuraAcademica}, a petición de las unidades académicas\footnote{ver \refElem{UnidadAcademica}} se envía el catálogo de profesores actualizado por parte de la \refElem{DCH}. Con esta infromación se inicia el proceso de Planeación de semestre y finalmente la generación de la \refElem{EstructuraAcademica} por Unidad.

		%-------------------------------------------
		%Diagrama de arquitectura
		
		\noindent La Figura \cdtRefImg{arq:AG-GP}{AG-GP Gestión de Profesores} muestra la interacción con los procesos que proporcionan la información necesaria para llevar a cabo sus funciones.

		\Pfig[0.8]{pgp/imagenes/arquitecturadeprocesos.jpg}{arq:AG-GP}{AG-GP Arquitectura General de Gestión de Profesores}} {AG-GP}

\end{Arquitectura}

%========================================================
%Interacción
%-----------------------------------------------

\begin{ADescripcion}
	\item \textbf{PG-GP1 Unidad Académica}. La \refElem{UnidadAcademica} continuamente envía la información y actualización de la plantilla de los Profesores/Docentes (cambio de adscripción, actualización de estado, solicitud de sabatico, solicitud de basificación, etc) para mantener al día el catálogo de profesores de la Unidad.
	
	\item \textbf{PG-GP2 Capital Humano}. Los cambios o actualizaciones solicitadas por la \refElem{UnidadAcademica}, se procede a revisar su pertinencia y si es procedente se confirman los cambios y se envía para su procesamiento interno, en caso contrario se regresa una notificación de que no procede el cambio. 
	
	\item \textbf{PG-GP3 Gestión de Profesores}. Para iniciar todo el proceso de estructura académica se realiza primero una planeación del semestre, para esto la \refElem{UnidadAcademica} solicita la plantilla actualizada a \refElem{CapitalHumano} con la que realizará la estimación de horarios y grupos. 
	
	\item \textbf{PG-GP4 Consulta de Profesores}. La \refElem{DCH} recibe la solicitud de la \refElem{UnidadAcademica}, realiza una consulta y filtra a los docentes de la \refElem{UnidadAcademica} y los regresa en forma de un Catálogo de Profesores actualizado con la categoria de cada docente.  
	
	\end{ADescripcion}

%\begin{PDescripcion}		
%\end{PDescripcion}