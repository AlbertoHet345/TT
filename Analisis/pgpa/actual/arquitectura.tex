%========================================================
%Arquitectura de Proceso
%========================================================

%========================================================
%Revisión
%-------------------------------------------

% \UCccitem{Versión}{1}
% \UCccsection{Análisis de Procesos }
% \UCccitem{Autor}{nombreAutor}
% \UCccitem{Evaluador}{nombreEvaluador}
% \UCccitem{Prioridad}{Alta} %Alta, media, baja
% \UCccitem{Estatus}{} %Edición, Terminado, Corrección, Aprobado 
% \UCccitem{Complejidad}{Alta} %Alta, Media, Baja
% \UCccitem{Volatilidad}{Alta} %Alta, Media, Baja
% \UCccitem{Madurez}{Media}  %Alta, Media, Baja
% \UCccsection{Control de cambios}
% \UCccitem{Versión 0}{
% \begin{UClist}
% \RCitem{ Pxn T1}{Corregir la ortografía}{\DONE}
% \TODO es para solicitar un cambio, \TOCHK Es para informar que se atendió el TODO, \DONE Es para indicar que el evaluador reviso los cambios.
% \end{UClist}
%}

%========================================================
% Descripción general del proceso
%-----------------------------------------------
\begin{Arquitectura}{AG-GPA}{Arquitectura General del Proceso de Gestión de Programas Académicos} {
		
		%-------------------------------------------
		%Descripción
		El proceso de Gestión de Programas Académicos incluye aspectos de diseño y rediseño, los cuales requiren la aprobación por parte de la \refElem{DES} y por parte del \refElem{ConsejoGeneralConsultivo} del IPN. La \refElem{DAE} es la respondable de su registro. 
		
		Éste proceso proporciona como salida, la definición de las unidades de aprendizaje\footnote{ver \refElem{UnidadDeAprendizaje}} que se deben cursar para cumplir con un plan de estudios.
		
		%-------------------------------------------
		%Diagrama de arquitectura
		\noindent La Figura \cdtRefImg{arq:AG-GPA}{AG-GPA Arquitectura General del Proceso de Gestión de Programas Académicos} muestra la interacción con los procesos que proporcionan la información necesaria para que se pueda registrar un programa de estudios.

		\Pfig[1.0]{pgpa/imagenes/ArquitecturaPP-GPAGestionProgramasAcademicos}{arq:AG-GPA}{AG-GPA Arquitectura General del Proceso de Gestión de Programas Académicos}
		
	}{AG-GPA}

\end{Arquitectura}

%========================================================
%Interacción
%-----------------------------------------------

\begin{ADescripcion}

	\item \textbf{Diseño/Rediseño de Programas Académicos}.  obedece a necesidades sociales detectadas por las Unidades Académicas\footnote{ver \refElem{UnidadAcademica}}, con el objeto de satisfacer la demanda de la industria e investigación. Los Diseñor o rediseños son generados por las Unidades Académicas con base en metodologías definidas en el IPN y enviadas a la \refElem{DES} para  su evaluación.

	\item \textbf{Aprobación de Programas Académicos por \refElem{DES}}. Éste proceso es el encargado de evaluar las propuestas de diseño y rediseño, genera las correcciones pertinentes o dictamina su aprobación. Cada uno de los programas académicos aprobados deben ser  enviados al \refElem{ConsejoGeneralConsultivo} , para realizar la evaluación final de las propuestas para que el \refElem{ProgramaAcademico} pueda ser impartido en la \refElem{UnidadAcademica} correspondiente y registrado en el \refElem{SAES}.

	\item \textbf{Aprobación de Programas Académicos por \refElem{ConsejoGeneralConsultivo}}. Es la evaluación generada por el \refElem{ConsejoGeneralConsultivo}, la cual determina su publicación en la Gaceta.

	\item \textbf{Registro de Programas Académicos\footnote{ver \refElem{ProgramaAcademico}} en \refElem{SAES}}.  El proceso de estructura académica permite definir que unidades de aprendizaje\footnote{ver \refElem{UnidadDeAprendizaje}} serán impartidas, para que este pueda operar requiere tener publicados los progamas académicos\footnote{ver \refElem{ProgramaAcademico}} en el sistema de gestión escolar, actualmente el \refElem{SAES}.

	\item \textbf{Gestión de Estructura Académica}. Éste proceso consume la información de cada uno de los programas académicos aprobados, para que con la información complementaria respecto a profesores, horarios e infraestructura, cada unidad obtenga su \refElem{EstructuraAcademica}.  
	
\end{ADescripcion}

%\begin{PDescripcion}		
%\end{PDescripcion}