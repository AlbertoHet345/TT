\chapter{Proceso fortalecido Gestión de Inscripciones}
\hypertarget{chapter:PFGI}{}
\section{AF-GI Arquitectura}

	La figura \cdtRefImg{afpi:procGral}{AF-GI Arquitectura del Proceso Fortalecido de Gestión de Inscripciones} muestra el proceso general modificado. Los cambios están centrados en la incorporación del \refElem{Calmecac}, la mejora en la comunicación entre las áreas de la \refElem{DAE}, la actualización mediante el Calmécac del \refElem{CalendarioAcademico}, los Programas Académicos\footnote{ver \refElem{ProgramaAcademico}} y las Estructuras Académicas\footnote{ver \refElem{EstructuraAcademica}}. De manera específica los cambios son:
\begin{Citemize}
	\item El Calmécac permitirá cargar la información de los calendarios académicos de todas las modalidades\footnote{ver \refElem{Modalidad}} para facilitar la supervisión y control de las actividades.
	\item Al fortalecer los procesos de Programa Académico (ver capítulo~\hyperlink{chapter:PFGPA}{Proceso Fortalecido de Gestión de Programas Académicos}) y Estructura Académica (ver capítulo~\hyperlink{chapter:PFEA}{Proceso Fortalecido de Estructura Académica}), se facilita la inscripción al tener la información actualizada en el Calmécac.
	\item Integrará mediante servicios web a los sistemas Calmécac y el sistema del departamento de Informática.
	\item Facilitará mediante el Calmécac la asignación de turno y grupo con algunos criterios establecidos.
\end{Citemize}

	La descripción detallada de estas mejoras se encuentra en la sección~\ref{sec:PF-IN:validacion}.

	\Pfig[1]{procesosfortalecidos/pin/procGral}{afpi:procGral}{AF-GI Arquitectura del Proceso Fortalecido de Gestión de Inscripciones.Para poder leer este diagrama ir a la seccion \ref{section:CodigoColores}}

