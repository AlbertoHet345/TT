\subsection{Valiación de Cambios del Proceso de Registro de Evaluaciones}
\label{sec:PF-GE:validacion}
%Perfil actual: Se espera poder poner el área, donode no se tenga certeza de el área especifica se puede poner de forma general.
%Tipo: básicamente son tres Cambio, Nuevo o Eliminar 

\hrule
\vspace{0.2cm}
\begin{Cdescription}
	\item[Subproceso:] Registro de calificaciones y Modificación de calificaciones.	
	\item[Situación actual:]En el proceso que comprende el registro de calificaciones no existe un mecanismo que autentique las identidades de quienes registran y modifican las calificaciones,  y mucho menos garantiza el no repudio.  
	\item[Perfil actual:] \refElem{UnidadAcademica}. - \refElem{DAE}.
	\item[Solución propuesta:] Se propone que el CALMÉCAC identifique y autentique a los actores responsables del registro y modificación de las calificaciones, haciendo uso de la firma electrónica como mecanismo de seguridad.
	\item[Perfil propuesto:] \refElem{UnidadAcademica}. - \refElem{DAE}.
	\item[Tipo:] Cambio.\\
\end{Cdescription}

\hrule
\vspace{0.2cm}
\begin{Cdescription}
	
	\item[Subproceso:] Registro de calificaciones
	\item[Situación actual:]No existe una diferencia entre las diferentes modalidades escolares.
	\item[Perfil actual:] \refElem{UnidadAcademica}.
	\item[Solución propuesta:] Se propone que el CALMÉCAC permita el registro de calificaciones diferenciando entre las distintas modalidades escolares y sus periodos correspondientes.
	\item[Perfil propuesto:] \refElem{UnidadAcademica}.
	\item[Tipo:] Cambio.\\
	
\end{Cdescription}

\hrule
\vspace{0.2cm}
\begin{Cdescription}
	
	\item[Subproceso:]Solicitar corrección especial.
	\item[Situación actual:]El proceso de corrección especial consta de una solicitud por escrito que contiene la petición del cambio de calificación junto con las firmas autógrafas del profesor, jefe de gestión académica, subdirector académico, Director de la UA y del supervisor de la DAE que valida el cambio, hasta el momento no hay un folio único para localizar estos documentos de manera rápida y eficiente.
	\item[Perfil actual:] \refElem{UnidadAcademica} \refElem{DES}.
	\item[Solución propuesta:] Se propone que el CALMÉCAC genere un folio digital para estas solicitudes único e irrepetible, así como firmar la solicitud de manera digital por todos los actores involucrados.
	\item[Perfil propuesto:] \refElem{UnidadAcademica} \refElem{DES}.
	\item[Tipo:] Cambio.\\
	
\end{Cdescription} 

\hrule
\vspace{0.2cm}
\begin{Cdescription}
	\item[Subproceso:]Registro de ETS.
	\item[Situación actual:]El proceso de registrar ETS consiste en que los alumnos que adeudan alguna materia deben de hacer su tramite de manera presencial ya que el sistema actual no detecta a los alumnos candidatos a presentar este examen  
	\item[Perfil actual:] \refElem{UnidadAcademica}.
	\item[Solución propuesta:] Se propone que el CALMÉCAC permitira la inscripción a ETS con base al historial académico de los alumnos, permitiéndole a esté decidir cual o cuales son los que presentara.
	\item[Perfil propuesto:] \refElem{UnidadAcademica}.
	\item[Tipo:] Cambio.\\
	
\end{Cdescription} 
%
%\begin{table}[htbp]
%\begin{center}
%\begin{tabular}{|p{2.2cm}|p{5.2cm}|p{2.2cm}|p{2.2cm}|p{1.2cm}|p{1.2cm}|}
%\hline
%Subproceso & Situación Actual &Perfil Actual &Solución propuesta & Perfil Propuesto & Tipo

%	\\ \hline
%		Registro del \refElem{Kardex}&
%		
%		El jefe del \refElem{DepartamentoDeGestionEscolar} junto con un supervisor de la \refElem{DAE} son los encargados de registrar las calificaciones y avalarlas respectivamente en el \refElem{Kardex} de manera manual. &
%		\refElem{UnidadAcademica}&
%		
%		Se propone que el CALMÉCAC lleve el registro del \refElem{Kardex} de manera digital implementando firma electrónica como mecanismo de seguridad. &
%		
%		\refElem{UnidadAcademica}&
%		
%		Cambio

%	\\ \hline
%		Inicio del periodo académico&
%		
% 		Para dar inicio al registro de calificaciones el jefe del \refElem{DepartamentoDeGestionEscolar} debe insertar  una fecha de inicio y fin en el \refElem{SAES} del periodo de devaluaciones en las diversas modalidades con base en el calendario escolar emitido por la \refElem{DAE}.&
% 		
% 		\refElem{UnidadAcademica}&
% 		
% 		Se propone que el CALMÉCAC de inicio al registro de calificaciones de manera automática basándose en el calendario escolar emitido por la \refElem{DAE}.&
% 		
% 		\refElem{DAE}&
% 		
% 		Cambio
% 
%
%\end{tabular}
%\caption{Tabla muy sencilla.}
%\label{tabla:sencilla}
%\end{center}
%\end{table}