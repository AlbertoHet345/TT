\chapter{Proceso Fortalecido Gestión de Evaluaciones}
\hypertarget{chapter:AF-GE}{}
\section{AF-RE Arquitectura}


La figura \cdtRefImg{afpgre:procGral}{AF-RE Registro de Evaluaciones} muestra el proceso general modificado. Los cambios están centrados en la incorporación del \refElem{Calmecac}, la mejora estará en el registro de calificaciones implementando la firma digital como mecanismo de seguridad para verificar la autenticidad y garantizar el no repudio de cualquier calificación, así como un folio digital para el subproceso de modificación de calificaciones como un identificador único en el Instituto Politécnico Nacional para el acta que se generará, de esta manera se simplifica la tarea de validar modificaciones por parte de la  \refElem{DAE} ya que sólo se necesitará su firma digital para autorizar un cambio de calificación. Se proponen los siguientes cambios:
\begin{Citemize}
    \item El Calmécac deberá incorporar la firma digital como mecanismo de seguridad para el registro y modificación de calificaciones para verificar la autenticidad y garantizar el no repudio de éstas.
    \item El Calmécac deberá incorporar el folio digital como mecanismo de seguridad y control de las actas que se generan al solicitar cambios de calificaciones ya que garantizará un identificador único e irrepetible.
\end{Citemize}

La descripción detallada de estas mejoras se encuentra en la sección~\ref{sec:PF-GE:validacion}.
\pagebreak
\Pfig[0.9]{procesosfortalecidos/pgre/imagenes/AGF.png}{afpgre:procGral}{AF-RE Arquitectura del proceso fortalecido de registro de evaluaciones.Para poder leer este diagrama ir a la sección \ref{section:CodigoColores}}
\pagebreak
