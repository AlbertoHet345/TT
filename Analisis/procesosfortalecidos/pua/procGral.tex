
\chapter{Proceso Fortalecido de Estructura Académica}
\hypertarget{chapter:PFEA}{}

\section{AF-EA Arquitectura}


%\begin{Proceso}{PF.X}{Nombre del proceso general}{
%		Anote la descripción del proceso general, un párrafo que describe brevemente: cuando inicia el proceso, su secuencia principal de actividades y productos principales.
%	}
%	{PG.X}
%	\PRccsection{Datos para control interno}
%	\PRccitem{Versión}{1}
%	\PRccitem{Autor}{Nombre completo del responsable del proceso}
%	\PRccitem{Evaluador}{Nombre completo del evaluador}
%	\PRccitem{Prioridad}{Alta/Media/Baja}
%	\PRccitem{Estatus}{Terminado/Corrección/Aprobado}
%	\PRccitem{Complejidad}{Alta/Media/Baja}
%	\PRccitem{Volatilidad}{Alta/Media/Baja}
%	\PRccitem{Madurez}{Alta/Media/Baja}
%	\PRccsection{Control de cambios}
%	\PRccitem{Versión 0}{
%			\begin{Titemize}
%				%\RCitem{ PC1}{Corregir la ortografía}{\DONE}
%				%\TODO es para solicitar un cambio \TOCHK Es para informar que se atendió el TODO(ya hizo las correcciones),\DONE Es para indicar que el e valuador reviso los cambios.
%			\end{Titemize}
%	}
%	\PRitem{Participantes}{
%		 Liste los participantes en el proceso, ya sean: áreas, organos colegiados o individuos, utilice el comando \refElem{idDelActor}.
%	}
%	\PRitem{Objetivo}{
%		Escriba un resumen a manera de objetivo (Que-Caracterisitica-para que) que englobe las responsabilidades relacionadas con el proceso y los problemas que resuelve.
%	}
%	\PRitem{Interrelación con otros procesos}{	
%		\begin{Titemize}
% 			\Titem Liste los procesos con que se enlaza la operación del proceso actual.
%		\end{Titemize}
%	}
%	\PRitem{Entradas}{
%		\begin{Titemize}
% 			\Titem Liste los datos, formatos o insumos que se requieren como entradas a lo largo de este procesos.
%		\end{Titemize}		
%	}
%	\PRitem{Salidas}{
%		\begin{Titemize}
% 			\Titem 	Liste los datos, formatos o insumos que se requieren como salidas o productos a lo largo de este procesos.
%		\end{Titemize}		
%	}
%	
%\end{Proceso}

	La figura \cdtRefImg{afpua:procGral}{AF-EA Estructura Académica} muestra el proceso general de la gestión de aprobación y autorización de la \refElem {EstructuraAcademica}. Los cambios están centrados en la incorporación del \refElem{Calmecac} y la alimentación de información del mismo hacia otros sistemas administrados por la \refElem{DCH}.
	
	

\begin{Citemize}
	\item Se propone que el Calmécac provea de información a los sistemas \refElem{SIIEE} (SIIEE) y \refElem{SRN} (SRN)a través de Servicios Web, para compartir información de la estructura de los profesores\footnote{Ver \refElem{Profesor}} de base e interinato continuo y profesores de sustitución respectivamente.
%	\item Se pretende habilitar una comunicación entre el Calmécac y los sistemas que administra la DCH para la compartición de información en ambos sentidos.
    \item Se busca eliminar el trabajo repetido de capturar la estructura académica en los sistemas SIIEE, SRN y \refElem {SAES} (SAES), al realizar una sóla vez la captura de la estructura en el Calmécac y compartir dicha información con las instancias que lo requieran.
     \item Se propone que el Calmécac permita adjuntar el soporte documental de los profesores que no están cubriendo la carga máxima de acuerdo a su \refElem{DictamenDeCategoria}.
    
\end{Citemize}

%	La descripción detallada de estas mejoras se encuentra en la sección~\ref{sec:PF-IN:validacion}.
	\pagebreak
	
	\Pfig[1]{procesosfortalecidos/pua/procGral}{afpua:procGral}{AF-EA Arquitectura del proceso fortalecido de Estructura Académica. Para poder leer este diagrama ir a la sección \ref{section:CodigoColores}}