
\begin{Proceso}{PG.X}{Nombre del proceso Específico}{
		Anote la descripción del proceso específico, un párrafo que describe brevemente: cuando inicia el proceso, su secuencia principal de actividades y productos principales.
	}
	{PE.X.X}
	\PRccsection{Datos para control interno}
	\PRccitem{Versión}{1}
	\PRccitem{Autor}{Nombre completo del responsable del proceso}
	\PRccitem{Evaluador}{Nombre completo del evaluador}
	\PRccitem{Prioridad}{Alta/Media/Baja}
	\PRccitem{Estatus}{Terminado/Corrección/Aprobado}
	\PRccitem{Complejidad}{Alta/Media/Baja}
	\PRccitem{Volatilidad}{Alta/Media/Baja}
	\PRccitem{Madurez}{Alta/Media/Baja}
	\PRccsection{Control de cambios}
	\PRccitem{Versión 0}{
			\begin{Titemize}
				%\RCitem{ PC1}{Corregir la ortografía}{\DONE}
				%\TODO es para solicitar un cambio \TOCHK Es para informar que se atendió el TODO(ya hizo las correcciones),\DONE Es para indicar que el e valuador reviso los cambios.
			\end{Titemize}
	}
	\PRitem{Participantes}{
		 Liste los participantes en el proceso (solo aquellos que realizan las tareas en el proceso), especifique a nivel de personas indicando el puesto y área a la que pertenece. Utilice el comando \refElem{idDelActor}.
	}
	\PRitem{Objetivo}{
		Escriba un resumen a manera de objetivo (Que-Caracterisitica-para que) que englobe las responsabilidades relacionadas con el proceso y los problemas que resuelve.
	}
	\PRitem{Interrelación con otros procesos}{	
		\begin{Titemize}
 			\Titem Liste los procesos con que se enlaza la operación del proceso actual.
		\end{Titemize}
	}
	\PRitem{Proveedores}{
		 Liste los participantes que proveen información al proceso, indique área y puesto o cargo, utilice el comando \refElem{idDelActor}.
	}
	\PRitem{Entradas}{
		\begin{Titemize}
 			\Titem Liste los datos, formatos o insumos que se requieren como entradas a lo largo de este procesos.
		\end{Titemize}		
	}
	\PRitem{Consumidores}{
		 Liste los participantes que reciben información o los resultados del proceso, indique área y puesto o cargo, utilice el comando \refElem{idDelActor}.
	}
	\PRitem{Salidas}{
		\begin{Titemize}
 			\Titem 	Liste los datos, formatos o insumos que se requieren como salidas o productos a lo largo de este procesos.
		\end{Titemize}		
	}
	\PRitem{Precondiciones}{
		\begin{Titemize}
 			\Titem Liste las actividades, productos o condiciones que deben ocurrir o cumplirse antes de iniciar el proceso.
		\end{Titemize}
	}
	\PRitem{Postcondiciones}{
		\begin{Titemize}
 			\Titem Liste los productos o condiciones que se logran o cambian al terminar el proceso de forma correcta.
		\end{Titemize}
		
	}
	\PRitem{Frecuencia}{
		Periódico, programado, por evento.
		% Periódico: Cada cierto tiempo: diario, semanal, anual, etc.
		% Programado: Alguien en algún momento establece la fecha.
		% Eventual: Cada que ocurre un evento que no se puede prever ni programar.
	}
	\PRitem{Tipo}{
		Mejora Continua / Operación / Soporte
	}
	% Operacion: Proceso asociado a las actividades propias de la operación del RENIECYT.
	% Mejora continua: Porcesos asociados a actividades de mejora continua del proceso actual.
	% Soporte: Procesos asociados a actividades indirectas necesarias para operar el RENIECYT.
	\PRitem{Áreas de oportunidad}{
		Liste los problemas que este nuevo proceso pretende resolver o atacar. 
	}
\end{Proceso}

	La figura \cdtRefImg{arq:AG-RE}{AG-RE Registro de Evaluaciones} muestra los procesos que componen el presente proceso general.

%		\Pfig[0.8]{proceso/imagenes/procesoGeneral}{pg:procGral}{Proceso general PGXX- XXXXXX}


%Descripción de procesos
\begin{PDescripcion}
	\Ppaso[\PSubProceso] \textbf{Nombre del Proceso:} Descripción del proceso o tarea...
	\Ppaso[\itarea] \textbf{Nombre de la Tarea:} Descripción de la tarea...
	\Ppaso[\Einicio] \textbf{Nombre de la Tarea:} Descripción de la tarea...
\end{PDescripcion}


%Factores criticos
\begin{FCDescripcion}
	\FCpaso Listar los factores críticos en este proceso
\end{FCDescripcion}








