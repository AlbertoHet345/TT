\chapter{Proceso Fortalecido Registro Horarios}
%\label{ch:PF-RH}
\hypertarget{chapter:PFRH}{}
\section{AF-RH Arquitectura}
%\begin{Proceso}{AF-GI}{Arquitectura del proceso fortalecido Gestión de Inscripciones}{
%		%El Proceso de Ingreso corresponde al mecanismo por medio del cual un \refElem{Aspirante}, que realizó todos los trámites correspondientes al \textbf{Proceso de Admisión} y fue seleccionado para pertenecer a una unidad académica se convierte en \refElem{Alumno} del Instituto. El alumno es el centro de la vida académica y el principal usuario de todos los servicios que ofrece el Instituto. 
%		}
%	{PG.X}
%%	\PRccsection{Datos para control interno}
%%	\PRccitem{Versión}{1}
%%	\PRccitem{Elaboró}{Francisco Isidoro Mera Torres}
%%	\PRccitem{Supervisó}{Ulises Vélez Saldaña}
%%	\PRccitem{Prioridad}{Alta}
%%	\PRccitem{Estatus}{Corrección}
%%	\PRccitem{Complejidad}{Media}
%%	\PRccitem{Volatilidad}{Media}
%%	\PRccitem{Madurez}{Media}
%	\PRsection{Atributos del proceso}
%	\PRitem{Participantes}{
%		\refElem{AbogadoGeneral}, \refElem{Alumno}, \refElem{Aspirante},\refElem{Calmecac}, \refElem{ComisionEspecial}, \refElem{DepartamentoDeRegistroYSupervisionEscolar}, \refElem{SubdireccionAcademica}
%	}
%	\PRitem{Objetivo}{
%		Asignar aspirantes a programas y unidades académicas otorgándoles una boleta y una identificación institucional.
%	}
%	\PRitem{Interrelación con otros procesos}{	
%		\begin{Titemize}
% 			\Titem Estructura Académica
%		\end{Titemize}
%	}
%	\PRitem{Entradas}{
%		\begin{Titemize}
% 			\Titem Estructura Académica.
% 			\Titem \refElem{HojaDeResultadoDelExamenDeAdmision}.
% 			\Titem Aspirantes Aceptados.
% 			\Titem Aspirantes Admitidos.
% 			\Titem \refElem{CalendarioAcademico}.
% 			\Titem \refElem{Convocatoria} y Calendario de Admisión.
% 			\Titem Programas Académicos Vigentes.
% 			\Titem Expedientes de Aspirantes.
%		\end{Titemize}
%	}
%	\PRitem{Salidas}{
%		\begin{Titemize}
% 			\Titem \refElem{Horario}.
% 			\Titem \refElem{Boleta} y/o \refElem{Preboleta}.
% 			\Titem \refElem{CredencialDeAlumno}.
% 			\Titem Notificación de documentación falsa o alterada.
%		\end{Titemize}		
%	}
%	
%\end{Proceso}

La figura \cdtRefImg{afrh:PF-RHArquitectura}{AF-RH Arquitectura del Proceso Fortalecido de Registro de Horarios} muestra el proceso general modificado. La generación y Registro de Horarios que se muestra en el proceso general está enfocada en mejorar la comunicación entre el \refElem{Calmecac} y su interacción con las Unidades Académicas, y las áreas responsables de proveer la información correspondiente de profesores (\refElem{CapitalHumano}), unidades de aprendizaje\footnote{ver \refElem{UnidadDeAprendizaje}}, e infraestructura académica, que serán utilizados para generar la propuesta de horarios, motivo por el cual se proponen las siguientes mejoras para el Calmécac:

\begin{Citemize}
	\item Generará la información estadística preprocesada para facilitar el análisis de la oferta de número de grupos a aperturar.
	\item Permitirá copiar la estructura académica utilizada en un periodo escolar similar para agilizar el proceso de generación de estructura académica, permitiendo realizar las modificaciones pertinentes sobre la misma.
	\item Realizará la detección de traslapes al momento de asignar los espacios académicos, unidades de aprendizaje, profesores.
	\item Identificará que profesores cubren con la totalidad de la carga académica con base en su categoría.
	\item Detectará que profesores sobrepasan su carga académica con base en su categoría.
	\item Determinará que unidades de aprendizaje puede impartir un profesor con base en la información histórica.
	
	
%	\item Genere la información estadística preprocesada para facilitar el análisis de la oferta de número de grupos a aperturar.
%	\item Permita copiar la estructura académica utilizada en un periodo escolar similar para agilizar el proceso de generación de estructura académica, permitiendo realizar las modificaciones pertinentes sobre la misma.
%	\item Realice la detección de traslapes al momento de asignar los espacios académicos, unidades de aprendizaje, profesores.
%	\item Identifique que profesores cubren con la totalidad de la carga académica con base en su categoría.
%	\item Detecte que profesores sobrepasan su carga académica con base en su categoría.
%	\item Determine que unidades de aprendizaje puede impartir un profesor con base en la información histórica.


	
\end{Citemize}

La descripción detallada de estas mejoras se encuentra en la sección~\ref{sec:PF-RH:validacion}.

\pagebreak

\Pfig[1]{procesosfortalecidos/phr/PF-RHArquitecturaVF}{afrh:PF-RHArquitectura}{AF-GI Arquitectura del Proceso Fortalecido de Registro de Horarios. Para poder leer este diagrama ir a la sección \ref{section:CodigoColores}}


%\begin{Proceso}{PF.}{Registro de Horarios}{
%		Anote la descripción del proceso general, un párrafo que describe brevemente: cuando inicia el proceso, su secuencia principal de actividades y productos principales.
%	}
%	{PG.X}
%	\PRccsection{Datos para control interno}
%	\PRccitem{Versión}{1}
%	\PRccitem{Autor}{Nombre completo del responsable del proceso}
%	\PRccitem{Evaluador}{Nombre completo del evaluador}
%	\PRccitem{Prioridad}{Alta/Media/Baja}
%	\PRccitem{Estatus}{Terminado/Corrección/Aprobado}
%	\PRccitem{Complejidad}{Alta/Media/Baja}
%	\PRccitem{Volatilidad}{Alta/Media/Baja}
%	\PRccitem{Madurez}{Alta/Media/Baja}
%	\PRccsection{Control de cambios}
%	\PRccitem{Versión 0}{
%			\begin{Titemize}
%				%\RCitem{ PC1}{Corregir la ortografía}{\DONE}
%				%\TODO es para solicitar un cambio \TOCHK Es para informar que se atendió el TODO(ya hizo las correcciones),\DONE Es para indicar que el e valuador reviso los cambios.
%			\end{Titemize}
%	}
%	\PRitem{Participantes}{
%		 Liste los participantes en el proceso, ya sean: áreas, organos colegiados o individuos, utilice el comando \refElem{idDelActor}.
%	}
%	\PRitem{Objetivo}{
%		Escriba un resumen a manera de objetivo (Que-Caracterisitica-para que) que englobe las responsabilidades relacionadas con el proceso y los problemas que resuelve.
%	}
%	\PRitem{Interrelación con otros procesos}{	
%		\begin{Titemize}
% 			\Titem Liste los procesos con que se enlaza la operación del proceso actual.
%		\end{Titemize}
%	}
%	\PRitem{Entradas}{
%		\begin{Titemize}
% 			\Titem Liste los datos, formatos o insumos que se requieren como entradas a lo largo de este procesos.
%		\end{Titemize}		
%	}
%	\PRitem{Salidas}{
%		\begin{Titemize}
% 			\Titem 	Liste los datos, formatos o insumos que se requieren como salidas o productos a lo largo de este procesos.
%		\end{Titemize}		
%	}
%	
%\end{Proceso}
%
%	La figura \cdtRefImg{arq:AG-RE}{AG-RE Registro de Evaluaciones} muestra los procesos que componen el presente proceso general.
%
%%		\Pfig[0.8]{proceso/imagenes/procesoGeneral}{pg:procGral}{Proceso general PGXX- XXXXXX}
%
%
%%Descripción de procesos
%\begin{PDescripcion}
%	\Ppaso \textbf{Nombre del Proceso:} Descripción del proceso...
%\end{PDescripcion}
%
%
%%Factores criticos
%\begin{FCDescripcion}
%	\FCpaso Listar los factores críticos en este proceso
%\end{FCDescripcion}
