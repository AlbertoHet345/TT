%- - - - - - - - - - - - - - - - - - - - - - - - - -

\begin{AreaOportunidad}{PI-AO1}
	\item[Área:] \refElem{DAE}
	\item[Procesos:] Admisión e Ingreso.
	\item[Área de Oportunidad:] El mecanismo actual en el que el Departamento de Informática comunica al Departamento de Supervisión los aspirantes aceptados y su asignación a las Escuelas ({\em la cual se realiza mediante vistas de Oracle que deben ser consultadas manualmente y mediante ``scripts programados''.}) provoca que el trabajo para mantener al SAES actualizado implique mucho trabajo y tiempo.
	\item[Mejoras:] Mejorar dicha comunicación mediante la integración de ambos sistemas ({\em el de informática para el proceso de admisión y el Calmécac}) mediante Servicios Web, Formatos especializados que permitan automatizar el intercambio de información o la integración de ambas Bases de datos.
	\item[Impacto:] Medio.
	\item[Alcance:] Dentro del ámbito del proyecto Calmécac recae considerar la mejora de la comunicación entre estas áreas, pero no, adecuaciones o mejoras al Sistema de Admisión, las cuales de ser necesarias requerirán del apoyo del departamento de Informática para lograrlas.
\end{AreaOportunidad}

%- - - - - - - - - - - - - - - - - - - - - - - -

\begin{AreaOportunidad}{PI-AO2}
	\item[Área:] \refElem{DAE}.
	\item[Procesos:] Admisión e Ingreso.
	\item[Área de Oportunidad:] El mecanismo con el que el Departamento de Informática envía al Departamento de Supervisión las actualizaciones de boleta derivadas de la revisión de expedientes ({\em la cual se realiza mediante vistas de Oracle que deben ser consultadas manualmente y mediante ``scripts programados''}) ha provocado que se entreguen boletas asignadas a alumnos cuyo numero de boleta no ha sido cargado al SAES. Esto es importante debido a que el número de boleta es utilizado en el SAES para garantizar el acceso a todos los servicios que otorga el IPN.
	\item[Mejoras:] Mejorar dicha comunicación mediante la integración de ambos sistemas ({\em el de informática para el proceso de admisión y el Calmécac}) mediante Servicios Web, Formatos especializados que permitan automatizar el intercambio de información o la integración de ambas Bases de datos en menor tiempo, con menor esfuerzo y reducir la intervención del personal.
	\item[Impacto:] Medio.
	\item[Alcance:] Dentro del ámbito del proyecto Calmécac recae considerar la mejora de la comunicación entre estas áreas, pero no, adecuaciones o mejoras al Sistema de Admisión, las cuales de ser necesarias requerirán del apoyo del departamento de Informática para lograrlas.
\end{AreaOportunidad}
%- - - - - - - - - - - - - - - - - - - - - - - - 

\begin{AreaOportunidad}{PI-AO3}
	\item[Área:] \refElem{DAE}.
	\item[Procesos:] Ingreso e inscripción.
	\item[Área de oportunidad:] El mecanismo con el que las Escuelas inscriben en el SAES a los alumnos de nuevo ingreso es manual, uno por uno, llenando grupos bajo criterios que son ``usos y costumbres'' o ``políticas ampliamente utilizadas''. Algunos de ellos son: por genero, localidad, examen de conocimientos, si el alumno trabaja, etc. El problema es que no hay un criterio homogeneizado y el trabajo y tiempo que implica esta actividad.
	\item[Propuesta:] Desarrollar en el Calmécac un módulo que facilite esta tarea, para los criterios más utilizados o comunes, con el fin de homogeneizar criterios, reducir el error humano y reducir el esfuerzo invertido en esta tarea.
	\item[Impacto:] Medio.
	\item[Alcance:] Dentro del ámbito del proyecto Calmécac.
\end{AreaOportunidad}

%- - - - - - - - - - - - - - - - - - - - - - - - 
\begin{AreaOportunidad}{PI-AO4}
	\item[Área:] \refElem{DAE}.
	\item[Procesos:] Admisión e Ingreso.
	\item[Área de oportunidad:] Actualmente cuando un alumno ha infringido algún lineamiento que lo deja fuera del instituto de manera permanente, el SAES no cuenta con un mecanismo para consultar dicha situación o para prevenir un reingreso de dicho alumno.
	\item[Propuesta:] Que el Calmécac cuente con el registro de alumnos en esta situación para conocer el estado final de cada alumno, así como prevenir su reingreso.
	\item[Impacto:] Medio.
	\item[Alcance:] Dentro del ámbito del proyecto Calmécac.
\end{AreaOportunidad}

%- - - - - - - - - - - - - - - - - - - - - - - - 

\begin{AreaOportunidad}{PI-AO5}
	\item[Área:] \refElem{DAE}.
	\item[Procesos:] Admisión, Ingreso e Inscripción.
	\item[Área de oportunidad:] La estructura actual del SAES dificulta la operación cuando un alumno pertenece a más de un programa académico, plan de estudios o Unidad Académica. 
	\item[Propuesta:] En el Calmécac considerar que un alumno puede estar participando en más de un Programa académico, en mas de una modalidad, en más de un plan de estudios y en mas de una unidad académica.
	\item[Impacto:] Medio.
	\item[Alcance:] Dentro del ámbito del proyecto Calmécac.
\end{AreaOportunidad}

%- - - - - - - - - - - - - - - - - - - - - - - - 

\begin{AreaOportunidad}{PI-AO6}
	\item[Área:] \refElem{DAE}.
	\item[Procesos:] Asignación de boleta.
	\item[Área de oportunidad:] Los cambios implementados en la estructura y manejo de Número de Boleta dificulta su uso en las Unidades Académicas para la generación de reportes necesarios para planeación y atención de necesidades de otras áreas. Así como el hecho de que los alumnos cambien actualmente de Genero y de unidad académica, y que el numero de boleta se use como identificador, dificulta el cambio de boleta o la consistencia de la información. 
	\item[Propuesta:] A la par del desarrollo del Calmécac se propone revisar los datos cambiantes y potencialmente cambiantes del alumno y despojar al Numero de boleta de dicha información, así como enriquecer el Calmécac con los reportes que las Unidades académicas requieren para su operación. 
	\item[Impacto:] Alto.
	\item[Alcance:] Dentro del ámbito del proyecto Calmécac la definición y generación de reportes.
\end{AreaOportunidad}

